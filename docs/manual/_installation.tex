\chapter{Installation}
\label{chInstallation}

This section describes the installation procedure for the RoboCup 3D
Simspark simulation server and for the \libbats library.

\section{RoboCup 3D Simspark Simulation Server}

For the latest installation instructions of the simulator, see the SimSpark project's wiki: \\
\url{http://simspark.sourceforge.net/wiki}\\
There you will find instructions for installation on various Linux
distributions, as well as for Windows and MacOS.

\section{libbats}

\begin{description}
\item[Dependencies] Install the necessary dependencies. For Ubuntu,
  and possibly other Debian based distributions, run:
\begin{verbatim}
$ sudo apt-get install libxml2-dev libsigc++-2.0-dev libgtkmm-2.4-dev
\end{verbatim}
  On Arch, run as root:
\begin{verbatim}
$ pacman -S libxml2 libsigc++ gtkmm
\end{verbatim}
  You can leave out {\tt gtkmm} if you don't want to build the GTK
  based debugger implementation.
\item[libbats] Install the latest version of {\tt libbats}:
  \begin{enumerate}
  \item Download the latest source code package release from: \\
    \url{https://github.com/sgvandijk/libbats/releases}
  \item Unpack and navigate to the source code directory:
\begin{verbatim}
$ tar xvzf libbats-x.y.z.tar.gz
$ cd libbats-x.y.z
\end{verbatim}
  \item Use CMake to configure, then make and install:
\begin{verbatim}
$ mkdir build && cd build
$ cmake ..
$ make
$ sudo make install
\end{verbatim}
  \end{enumerate}
\end{description}

\section{RoboViz}

RCSSServer3D comes with a monitor, RCSSMonitor3D, however it is rather
basic, both graphically as functionally. We suggest using RoboViz
instead, which offers much better visualization, and an advanced
debugging interface that is supported by \libbats. Installation
instructions for RoboViz can be found at:\\
\url{https://sites.google.com/site/umroboviz/}

\section{Testing}

To test whether everything is working correctly, start the simulator
with:
\begin{verbatim}
$ rcssserver3d
\end{verbatim}
This command should give you a greeting saying something similar to
the following:
\begin{verbatim}
rcssserver3d (formerly simspark), a monolithic simulator 0.6.5
Copyright (C) 2004 Markus Rollmann, 
Universität Koblenz.
Copyright (C) 2004-2009, The RoboCup Soccer Server Maintenance Group.

Type '--help' for further information
\end{verbatim}
plus some initialization output. This output will contain some
messages such as `{\tt ERROR: cannot find TextureServer}' and `{\tt
  ERROR: no FPSController found at}\\{\tt
  '/usr/scene/camera/physics/controller'}'; you can ignore these
messages.

Next start RoboViz. Change to the directory containing RoboViz' binary
and run:
\begin{verbatim}
./roboviz.sh
\end{verbatim}
After some time in which RoboViz connects to the server and
initializes its graphics a green field will appear with some lines and
two goals. Ta-da, you are successfully running the simulator! If
RoboViz reports 'Disconnected', you are not and something went wrong
with installing the simulator; see below for some troubleshooting
tips.

Finally start the example agent supplied with \libbats; in the
\libbats build directory execute:
\begin{verbatim}
$ cd examples/helloworld
$ ./helloworld
\end{verbatim}
If everything went well, an agent should appear, standing in the left
side of the field, waving its arms. You can also try another, more
advanced example agent; again, in the \libbats build directory, run:
\begin{verbatim}
$ cd examples/dribble
$ ./dribble
\end{verbatim}
This should start an agent which, when you start the game, walks to
the ball and dribbles it over to the opponent's goal.

\section{Troubleshooting}
\begin{enumerate}
\item To uninstall Simspark or RCSSServer3D, simply execute the
  following command within the respective build directory:
\begin{verbatim}
$ sudo make uninstall
\end{verbatim}
\item If compilation fails due to missing dependencies, as reported
  when running {\tt cmake} for the simulator or for
  \libbats, try looking up the required libraries using your
  distributions's package manager. For Ubuntu:
\begin{verbatim}
$ sudo apt-cache search <missing package>
\end{verbatim}
  For Arch:
\begin{verbatim}
$ pacman -Ss <missing package>
\end{verbatim}
\item Please refer to the SimSpark wiki for more information on how to
  install (or use) simspark, e.g. when using a different operating
  system:\\
  \url{http://simspark.sourceforge.net/wiki/index.php/Main_Page}.
\item Search through the sserver-three-d mailing list to see if there
  is a solution to your problem:\\
  \url{http://sourceforge.net/search/?type_of_search=mlists&group_id=24184}\\
  otherwise post it to the list (see
  \url{http://sourceforge.net/mail/?group_id=24184}).
\item For problems with \libbats, see the GitHub Issues section at: \\
  \url{https://github.com/sgvandijk/libbats/issues}.
\end{enumerate}

%%% Local Variables: 
%%% mode: latex
%%% TeX-master: "libbatsmanual"
%%% End: 
