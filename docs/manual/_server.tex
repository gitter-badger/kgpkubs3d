\chapter{SimSpark Simulation 3D Server}
\label{chServer}

The simulation is generated by the SimSpark Soccer Simulation Server. Agents can enter
the simulation by communicating with the soccer server using tcp sockets and a
hierarchical parentheses based communication protocol (think of LISP). The simulation maintains a
physically realistic (although at the moment noiseless) world, existing of a
soccer field and soccer(ro)bots. One agent program represents one soccer robot in the
simulation.

As of the World Championship of 2007 the simulation can only reliably run two agents per
team and still has some hiccups in the form of physics errors causing exploding body parts.
But we are confident that these will be resolved in the near future.

If you want to learn more you can take a look at the user manual for the server, but it is not quite finished yet:\\
\url{http://sourceforge.net/apps/mediawiki/sserver/index.php?title=Users_Manual}

\section{The Soccer Field}

Currently the actual size of the Soccer Field is unknown. In each corner resides a flag and on each side are two goal
posts. The size of the field is subjected to change and probably will not stay the same for
very long. Luckily the server notifies the agent of the size of the field at the initiation of communication.

\section{The Soccerbot}

The soccerbot is a humanoid robot, based on Alderbaran's Naobot.

%\section{The Agent Communication Protocol}

% Socket communication (bad communication)
% Communication protocol
% Field layout (flags and goalposts)
% Joints of the robot + constraints
% Monitor

\section{Starting the server and monitor}
You can use the startup script provided in the bats source code:
\begin{verbatim}
$ <location to bats source code>/libbats/util/starttrunkserver
\end{verbatim}
Or you could start the simspark server and monitor separately:
\begin{verbatim}
$ simspark &
$ rcssmonitor3d
\end{verbatim}
See next chapter for more information about the Simspark Simulation 3D Monitor.

