\chapter{Quick Start}
\label{chQuickstart}

\lstset{numbers=left,numberstyle=\scriptsize}

This chapter is intended to quickly get you started with creating an
agent using {\tt libbats}. By following these steps you will recreate
the simple Hello World example agent that is supplied with the
library. See the next chapters for more detailed information on the
modules that are used, or when you only need a small part of the
library, like communication with the simulation server.

\section{Setting Up}
\label{sec:setting-up}

First, we will set up the basic project structure for coding and
compiling your agent.

\begin{itemize}
\item Create a directory that will hold all files, called say '{\tt
    myagent}. We will refer to this as the source directory.
\item Create a directory in this new directory, called `{\tt cmake}',
  and copy the following files from the \libbats source directory into
  it: {\tt FindEigen3.cmake},{\tt FindSigC++.cmake}, and {\tt
    LibFindMacros.cmake}.

\begin{lstlisting}[float,caption={\tt CMakeLists.txt},label=cmakelists,frame=single]
cmake_minimum_required (VERSION 2.6)
project (myagent)

set(CMAKE_MODULE_PATH ${CMAKE_SOURCE_DIR}/cmake/)
find_package(Eigen3 REQUIRED)
find_package(LibXml2 REQUIRED)
find_package(SigC++ REQUIRED)
PKG_CHECK_MODULES(GTKMM gtkmm-2.4)

set(CMAKE_CXX_FLAGS "-std=c++0x")

include_directories(
  ${EIGEN3_INCLUDE_DIR}
  ${LIBXML2_INCLUDE_DIR}
  ${SigC++_INCLUDE_DIRS}
  ${LIBXMLXX_INCLUDE_DIRS}
  ${GTKMM_INCLUDE_DIRS}
)

add_executable(myagent
myagent.cc
main.cc
)

target_link_libraries(myagent
  bats
  ${LIBXML2_LIBRARIES}
  ${SigC++_LIBRARIES}
  ${GTKMM_LIBRARIES}
)
\end{lstlisting}

\item Create a file called {\tt CMakeLists.txt} in your source
  directory and fill it with the content of listing \ref{cmakelists}. This will do the following:
  \begin{description}
  \item[lines 1-2] Give some header info, such as CMake version and your project name.
  \item[lines 4-8] Use the files copied in the previous step to find
    and comfigure required libraries. If you didn't build libbats with
    GTK debugger support, remove line 8, as well as lines 17 and 29.
  \item[line 10] We need to tell the compiler that \libbats
    extensively uses C++11 (the 0x standard is used, because older
    compilers don't support more than that).
  \item[lines 12-18] Tell CMake where to find all library headers.
  \item[lines 20-23] List all source files belonging to your agent
    that need to be compiled.
  \item[lines 25-30] Tell CMake which libraries to link to.
  \end{description}
\item Copy the `{\tt xml}' directory fully from the \libbats source
  directory to your own source directory.
\end{itemize}

\newpage
\section{Coding Your Agent}
\label{sec:coding-your-agent}

The base of a {\tt libbats} agent is the {\tt HumanoidAgent}
class. This class initializes all parts of the library and supplies a
simple life cycle for your agent. So let's start by creating your own
agent class by extending {\tt HumanoidAgent}. Of course, we have to
create a constructor, and the {\tt HumanoidAgent} class requires that
your agent defines an {\tt init()} and a {\tt think()} method. Listing
\ref{codeMyagenthh} shows what your header file may look like.

\begin{lstlisting}[float,caption={\tt myagent.hh},label=codeMyagenthh,frame=single]
#ifndef MYAGENT_HH
#define MYAGENT_HH

#include <libbats/HumanoidAgent/humanoidagent.hh>

/** My first agent */
class MyAgent : public bats::HumanoidAgent
{
  /** Initialize agent */
  virtual void init();
  
  /** Think cycle */
  virtual void think();
  
public:

  /** The Constructor */
  MyAgent()
    : bats::HumanoidAgent("MyTeam", "xml/conf.xml")
  {  }
};

#endif
\end{lstlisting}

Now, what should these methods do?
\begin{description}
\item[{\tt MyAgent()}] The constructor should give some initialization
  information to the constructor of the base class {\tt
    HumanoidAgent}. At least the name of your team should be supplied,
  but you could also set some parameters such as the host address and
  port number to connect to. In this case, the path to a custom XML
  configuration file is given. See the following chapters and details
  in {\tt HumanoidAgent}'s class documentation for more information on
  these parameters.
\item[{\tt init()}] This method is called once after the agent is
  created, a connection to the simulator is established, and all parts
  of the library are initialized. You can use this to initialize your
  own things, like a formation module, movement generators, et cetera.
\item[{\tt think()}] Here is where you put your agent's 'brain'. After
  the agent is started and initialized, this method is called at every
  think cycle, 50 times per second. When the {\tt think()} method is
  called, new sensor information from the server is read and
  integrated in different modules, like the {\tt AgentModel}, the {\tt
    WorldModel} and the {\tt Localizer} (more on these later). In this
  method your agent should decide what to do and make sure actions for
  the current think cycle are sent to the server.
\end{description}

At the moment we don't have our own fancy modules yet, so the
constructor is empty. However, we do want our agent to do something
cool, so we will fill the {\tt think()} method as shown in listing
\ref{codeMyagentcc} to make him wave his arms at us. Let's look at
what all of this does.

\begin{lstlisting}[float,caption={\tt myagent.cc},label=codeMyagentcc,frame=single]
#include "myagent.hh"
#include <libbats/Clock/clock.hh>
#include <libbats/AgentModel/agentmodel.hh>
#include <libbats/Cerebellum/cerebellum.hh>
#include <libbats/Localizer/KalmanLocalizer/kalmanlocalizer.hh>
#include <libbats/Debugger/RoboVizDebugger/robovizdebugger.hh>

using namespace bats;
using namespace std;

void MyAgent::init()
{
  // You must tell libbats which flavor localizer
  // and debugger you are using
  SLocalizer::initialize<KalmanLocalizer>();
  SDebugger::initialize<RoboVizDebugger>();
}

void MyAgent::think()
{
  Clock& clock = SClock::getInstance();
  AgentModel& am = SAgentModel::getInstance();
  Cerebellum& cer = SCerebellum::getInstance();
  
  double t = clock.getTime();
  
  // Get current joint angles
  double angles[4];
  angles[0] = am.getJoint(Types::LARM1)->angle->getMu()(0);
  angles[1] = am.getJoint(Types::LARM2)->angle->getMu()(0);
  angles[2] = am.getJoint(Types::RARM1)->angle->getMu()(0);
  angles[3] = am.getJoint(Types::RARM2)->angle->getMu()(0);
  
  // Calculate target joint angles
  double targets[4];
  targets[0] = 0.5 * M_PI;
  targets[1] = 0.25 * M_PI * sin(t / 2.0 * 2 * M_PI) + 0.25 * M_PI;
  targets[2] = 0.5 * M_PI;
  targets[3] = -0.25 * M_PI * sin(t / 2.0 * 2 * M_PI) - 0.25 * M_PI;
  
  // Determine angular velocities
  double velocities[4];
  for (unsigned i = 0; i < 4; ++i)
    velocities[i] = 0.1 * (targets[i] - angles[i]);
  
  // Add joint movement actions to the Cerebellum
  cer.addAction(make_shared<MoveJointAction>(Types::LARM1, velocities[0]));
  cer.addAction(make_shared<MoveJointAction>(Types::LARM2, velocities[1]));
  cer.addAction(make_shared<MoveJointAction>(Types::RARM1, velocities[2]));
  cer.addAction(make_shared<MoveJointAction>(Types::RARM2, velocities[3]));
  
  // Tell the Cerebellum to send the actions to the simulator
  cer.outputCommands(SAgentSocketComm::getInstance());
}
\end{lstlisting}

\begin{description}
\item[lines 1-9] First include the header file of your agent class,
  here {\tt helloworldagent.hh}, and the header files of the modules
  that are used. All {\tt libbats} classes are in the {\tt bats}
  namespace, so in line 6 we import this namespace so we don't have to
  type {\tt bats::} all the time. The {\tt std} namespace is also
  imported for convenience.
\item[lines 11-17] This is the implementation of the {\tt init}
  method, which is called once at start-up. Here you should initialize
  all your modules. At the very least, you must tell {\libbats} which
  classes of {\tt Localizer} and {\tt Debugger} you want to use.
\item[lines 21-23] Here references to the used modules are
  requested. Most modules are so called \emph{singletons}, which means
  there is only one instance of each class. The {\tt Clock} and {\tt
    AgentModel} do what you probably already expect they do: they give
  the current time and a model of the agent's state. The {\tt
    Cerebellum} is named after the part of your brain that handles
  control and coordination of your movements and is used to actually
  do stuff, as you will see later on.
\item[line 25] Get the current time.
\item[lines 28-32] We want our agent to wave his arms, by moving his
  shoulder joints. To do this it is useful to know the current state
  of these joints. This sounds like a job for the {\tt AgentModel} and
  as you can see it is. The {\tt Types} class defines all sorts of
  handy types used by several modules.
\item[lines 35-39] Next we define the angles we want to move the
  joints to. Here a sinusoidal pattern is used to create a smooth,
  friendly waving behavior.
\item[lines 42-44] The agent is controlled by setting the angular
  velocities of its joints, so here these are calculated based on the
  current and target angles and a gain factor.
\item[lines 47-50] As mentioned earlier, the {\tt Cerebellum} is used
  to act. It is fed with actions, in this case joint movements, but it
  also controls the other actuators like speech and beaming.
\item[line 53] When the {\tt Cerebellum} has gathered all actions, it
  is time to send them to the simulation server. A {\tt SocketComm},
  in the form of the specialized {\tt AgentSocketComm}, is needed for
  this, which handles the actual complicated communication through
  sockets.
\end{description}

\newpage

\begin{lstlisting}[float,caption={\tt main.cc},label=codeMaincc,frame=single]
#include "myagent.hh"

int main()
{
  MyAgent agent;
  agent.run();
}
\end{lstlisting}

And that's it! Almost. The only thing left to do now is to create an
actual executable program that runs our agent. This is done by
defining the standard {\tt main()} method, creating an instance of the
agent class and tell it to run, as shown in listing
\ref{codeMaincc}. Now, configure and compile your agent, by running
`{\tt cmake . \&\& make}' in your source directory, fire up RCSSServer3d
and a monitor, run the `{\tt myagent}' binary and wave back at your
new friend!

%%% Local Variables: 
%%% TeX-master: "libbatsmanual"
%%% End: 
