\chapter{SimSpark Simulation 3D Monitor}
\label{chSimulator}
The SimSpark Simulation 3D monitor ''rcssmonitor3d`` is used to observe and control the actual Simspark Simulation 3D Server.
\section{Invocation}

You may use the following command line options while starting rcssmonitor3d:
%XXX Not all options seems to work in current version!
\begin{description}
\item[{\tt --help}] Print a help message and exit.
%\item[{\tt --port}] Specify the port number (default is 12001).
%\item[{\tt --server}] Specify the server host (default is 'localhost').
\item[{\tt --logfile}] Specify the logfile to read.
%\item[{\tt --msgskip}] Every but the nth message should be discarded (default is 1).
%\item[{\tt --texture}] Set the name of the texturefile to use.  If no absolute path is given the file is assumed to reside in your ~/.rcssserver3d/ directory.
\end{description}

Using the {\tt --logfile} flag you can run a {\tt monitor.log} file you made earlier, simply by running {\tt rcssmonitor3d --logfile monitor.log}.
The monitor client connects by default to ''127.0.0.1:3200``.

\section{Usage}
%XXX '?' Does not seem to work in current version:
%After starting the monitor there are a few special keys you can use to move and manipulate your viewport. To get the most reset values, press {\tt ?}. After pressing questionmark, the monitor will output a message like keystroke information described in table \ref{monitorkeyshelp}.

While rcssmonitor3d (the OpenGL GUI) is running, a few special keys can be used to move and manipulate the viewport. These are described in table \ref{tabmonitorkeyshelp}.

\begin{table}[placement]

\caption{Keybindings for rcssmonitor3d}

\label{tabmonitorkeyshelp}
\begin{tabular}[t]{|l|l|}

\hline
& General \\
\hline
q          & Quit the monitor \\
p          & Pause the simulation/logplayer \\
r          & Unpause the simulation/logplayer \\
n          & Toggle uniform numbers \\
1          & Toggle debug information \\
2          & Toggle two dimensional overview \\
v          & Toggle velocities \\
?          & Display keybindings \\
\hline
& Camera movement \\
\hline
c          & Toggle ball centered camera \\
w/s        & Move camera in/out \\
a/d        & Move camera left/right \\
+/-        & Move camera up/down \\
3          & Move camera behind left goal \\
4          & Move camera behind right goal \\
\emph{Arrow keys} & Move camera \\
\hline
& Simulation \\
\hline
k          & Kick-off left (start the game) \\
j          & Kick-off right \\
l          & Free kick left \\
r          & Free kick right \\
b          & (Playon) Drop the ball at its current position \\
\emph{SPACE}      & Toggle kick off side (side-random) \\
\hline
& Logplayer \\
\hline
f          & Toggle fast/realtime replay \\
b          & Backward replay \\
F          & Realtime replay \\
m          & Toggle single step mode for logplayer \\
$>$        & Move one step forward \\
\hline

\end{tabular}

\end{table}

